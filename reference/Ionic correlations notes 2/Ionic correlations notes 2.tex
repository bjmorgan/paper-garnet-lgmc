% \documentclass[aip,apl]{revtex4-1}
% \documentclass[aps,prl,preprint,groupedaddress]{revtex4-1}
\documentclass[aps,prl,twocolumn,superscriptaddress,reprint]{revtex4-1}
\usepackage{graphicx}
\usepackage[utf8]{inputenc}
% \usepackage[pdfusetitle]{hyperref}
\usepackage{amsmath}
\usepackage{siunitx}
\graphicspath{{Figures/}}
\newcommand{\Li}[1]{Li$_{#1}$}

\usepackage[normalem]{ulem}
\usepackage{color}
\newcommand{\note}[1]{\textcolor[rgb]{0.2, 0.7, 0.3}{\textsc{[#1]}}}
\newcommand{\todo}[1]{\textcolor[rgb]{0.7, 0.2, 0.2}{$=>$ #1}}
\newcommand{\theme}[1]{\textcolor[rgb]{0.2, 0.2, 0.7}{\underline{#1}}\\}

\newcommand{\Arrh}[2]{\exp\left(\frac{-#1}{#2kT}\right)}

\newcommand{\chem}[1]{\ensuremath{\mathrm{#1}}}
\newcommand{\m}[1]{\mathrm{#1}}
\newcommand{\e}[1]{\mathrm{e}^{#1}}
\renewcommand{\b}{$\beta$}
\newcommand{\expp}[1]{\exp\left(#1\right)}
\newcommand{\x}{\times}
\renewcommand{\o}{\bullet}
\renewcommand{\i}[1]{\textit{#1}}
\renewcommand{\u}[1]{\,\mathrm{#1}}
\renewcommand{\vec}[1]{\mathrm{\textbf{#1}}}
\newcommand{\abrac}[1]{\left<#1\right>}
\newcommand{\pbrac}[1]{\left(#1\right)}
\newcommand{\avg}[1]{\left|#1\right|}
\newcommand{\sbrac}[1]{\left[#1\right]}
\newcommand{\etal}{\emph{et al}.\ }
\newcommand{\tocite}{\textcolor[rgb]{1.0, 0.4, 0.3}{[?]}}


\include{amsmath}

\begin{document}

\title{TITLE}
\author{Benjamin J. Morgan}
\affiliation{Department of Chemistry, University of Bath, Claverton Down, Bath, BA2 7AY}

\date{\today}

% \begin{abstract}

% \end{abstract}

\maketitle

Consider a particle that jumps between sites on a regular lattice, with separation $a$, and average jump frequency $\nu$. After $m$ jumps the total displacement is
\begin{equation}
  \vec{r} = \vec{a}_1 + \vec{a}_2 + \vec{a}_3 + \ldots + \vec{a}_m
\end{equation}
giving a mean-squared displacement of
\begin{equation}
  \left<\left|\vec{r}\right|^2\right> = \left( \vec{a}_1 + \vec{a}_2 + \ldots + \vec{a}_m \right)\left(\ \vec{a}_1 + \vec{a}_2 + \ldots + \vec{a}_m \right)
\end{equation}
which can be collected into two sums; one over squared terms, and one over cross-terms:
\begin{eqnarray}
  \left<\left|\vec{r}\right|^2\right> & = & \sum_i^m \left|\vec{a}_i\right|^2 + 2\sum_i^{m-1}\sum_{j>i}^m\left|\vec{a}_i\cdot\vec{a}_j\right|. \\
  & = & ma^2 + 2\sum_i^{m-1}\sum_{j>i}^m\left|\vec{a}_i\cdot\vec{a}_j\right|.
\end{eqnarray}
The diffusion coefficient for this particle is given by the (a?) Einstein relation
\begin{eqnarray}
  D^* & = & \frac{\left<\left|\vec{r}\right|^2\right>}{6t}.\\
      & = & \frac{ma^2}{6t}f\\
      & = & \frac{\nu a^2}{6}f
\end{eqnarray}
where
\begin{equation}
  f = 1 + \frac{2}{ma^2}\sum_i^{m-1}\sum_{j>i}^m\left|\vec{a}_i\cdot\vec{a}_j\right|.
\end{equation}
$f$ is the correlation factor.
\note{this is called the microscopic (ref?) or tracer diffusion coefficient} \note{property of individual ions}
If the particle follows a random walk, the directions of each jump are independent (uncorrelated), and the dot products between different jumps sum to zero: $f=1$. This is the case for non-interacting particles, or in the dilute limit. If jump directions are correlated, as is the case for example for vacancy diffusion (where a mobile atom is most likely to jump back to its previous site), then $f\ne1$, and the diffusion coefficient will differ from the random walk value.

In contrast to the microscopic diffusion coefficient, the ionic conductivity is an ensemble property of all the ions in the system and depends on the net mean-squared displacement of charge. For a single species of ions the net displacement after $M$ jumps is:
\begin{equation}
  \vec{R} = \sum_i^M \vec{a}_i
\end{equation}
and the mean-squared net displacment is
\begin{eqnarray}
  \left<\left|\vec{R}\right|^2\right> & = & \sum_i^M \left|\vec{a}_i\right|^2 + 2\sum_i^{M-1}\sum_{j>i}^M\left|\vec{a}_i\cdot\vec{a}_j\right|. \\
  % & = & ma^2 + 2\sum_i^{m-1}\sum_{j>i}^m\left|\vec{a}_i\cdot\vec{a}_j\right|.
\end{eqnarray}
The dc ionic conductivity is given by a second Einstein relation
\begin{eqnarray}
  \sigma & = & \frac{q^2}{VkT}\frac{\left<\left|\vec{R}\right|^2\right>}{6t} \\
         & = & \frac{q^2 N}{VKT}\frac{\nu a}{6}f_I,
\end{eqnarray}
where
\begin{equation}
  f_I = 1 + \frac{2}{Ma^2}\sum_i^{M-1}\sum_{j>i}^M\left|\vec{a}_i\cdot\vec{a}_j\right|.
\end{equation}

\begin{equation}
  f_I = 1 + \frac{2}{Ma^2}\left( \sum_\alpha\sum_i^m\sum_{j>i}^m\right)
\end{equation}
$f_I$ is the (collective?) correlation factor, and depends on correlations between \emph{all} jumps: including sequential jumps made by the a single ion, and jumps made by different ions. The contributions from these two types of correlation can be seen by expanding the net displacement, $\vec{R}$, in terms of contributions, $\vec{r}$, made by individual ions.
\begin{equation}
  \vec{R} = \sum_\alpha^N \sum_i^{m_\alpha}\vec{r}_i^\alpha
\end{equation}
where the index $\alpha$ runs over all contributing ions.
\begin{eqnarray}
  \left<\left|\vec{R}\right|^2\right> & = & \sum_\alpha^N\sum_i^{m_\alpha} \left|\vec{a}_i^\alpha\right|^2 + \sum_\alpha^N\sum_\beta^N\sum_i^{m_\alpha}\sum_j^{m_\beta}\left|\vec{a}_i^\alpha\cdot\vec{a}_j^\beta\right|. \\
  & = & Ma^2 + 2\sum_\alpha^N\sum_i^{m_\alpha-1}\sum_{j>i}^{m_\alpha}\left|\vec{a}_i^\alpha\cdot\vec{a}_j^\alpha\right| + 2\sum_\alpha^N\sum_{\beta\ne\alpha}^N\sum_i^{m_\alpha}\sum_j^{m_\beta}\left|\vec{a}_i^\alpha\vec{a}_j^\beta\right|.
\end{eqnarray}



\bibliography{Bibliography}
\end{document}
  
